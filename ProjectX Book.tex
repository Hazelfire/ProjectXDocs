\documentclass{book}
\author{Sam Nolan}
\title{Programming with ProjectX}
\begin{document}
	\maketitle
	\tableofcontents
	\chapter{Introduction}
	\section{What is ProjectX?}
	ProjectX is both a game and an educational tool for learning the basics of programming. ProjectX can be played without even touching the programming aspects of the game as a social multiplayer game, But becomes much more interesting when you create content of your own through the use of the programming languages Lua and XML.
	
	ProjectX is a great tool for learning programming for the first time, as it covers a wide range of skill levels.
	
	\section{Game-Play and Rules}
	The game itself is an Real Time Strategy multiplayer game where the objective is to be the last man standing. Players connect to a server, and are spawned into an Arena. The controls are mainly touch based as the game also ports the Android and iOS platforms. Players can roam around a map with the objective of collecting resources to survive the harsh environment that they find themselves in. Players will find creatures and other features that can be interacted with. Other players can also be found and they can choose to team up or to fight, last player or team alive wins.
	
	The game imposes very few constraints about what content can be added to the game. Creature, tiles, interactions, items and much more can be added to the game to make the experience completely different. Changes that are made to the game on one device can be synced to another simply by playing on the same game.
	
	\section{Programming with ProjectX}
	ProjectX has 2 official languages, Lua and XML. These language are serve different purposes in the game. XML is used to represent objects in the game, whereas with Lua procedures can be specified. XML controls map generation, Items, Sprites and many more, whereas Lua controls Creature AI and interactions. The skills learnt can be applied to other programming languages such as Javascript, HTML or any other language that a person might want to learn.
	
	To keep things simple, what you can do is fairly limited. A programmer can only add \textit{content} to the game. That is, it is not possible to add anything that does not already exist. For instance, it is possible to create your own creature to be placed into the game, but it is not possible to add UI elements and change the rules of the game around. These are called \textit{features} and are added by the maintaining developers of the game. If you wish to see a feature in the game, you may contact Sam Nolan at NOL0055@bhs.vic.edu.au.
	
	All the code that controls how the game runs can be found in the \texttt{scripts} folder of the installation directory (for Windows this is usually found at \\ \texttt{C:\textbackslash Program Files(x86)\textbackslash ProjectX}). Most of the code has its own jurisdiction, as in, the code in each folder controls a certain aspect of the game, and no more. This is in contrast to most languages, where all code does not need to be in separate files to function. We decided to make ProjectX like this as to keep the code simple as possible.
	
	\section{Installation}
	The installation of ProjectX is reasonably straightforward. Windows is currently the only supported platform. We do have Android versions but the game is still in development and they have not been released yet. In the future, ProjectX should be supported on Linux, OSX, iOS, Android and Windows.
	
	Because the game is currently in development, receiving a copy of the game can only be done through directly asking Sam Nolan. His contact details can be found in the Developers section of Notes and Acknowledgments.  
	\chapter{Notes and Acknowledgments}
	\section{Developers}
	This section concerns the developers of ProjectX, contact details are provided for some.
	
	\paragraph{Programming}
	\begin{itemize}
		\item Sam Nolan: NOL0055@bhs.vic.edu.au
		\item Michael Smith
	\end{itemize}
	
	\paragraph{Art}
	\begin{itemize}
		\item Oscar Taylor
		\item Codeword Gaming
		\item Louis Evans
	\end{itemize}
	
	\paragraph{Music}
	\begin{itemize}
		\item Louis Evans
	\end{itemize}
\end{document}